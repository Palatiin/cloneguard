% File: src/chapter_1_introduction.tex
% Project: Monitoring and Reporting Tool for Cloned Vulnerabilities across Open-Source Projects
% Author: Matus Remen (xremen01@stud.fit.vutbr.cz)
% Description: Chapter 1 - Introduction

\chapter{Introduction}
  Vulnerabilities in software can have serious consequences, including reputation damage,
  financial losses, or even loss of life in the case of critical infrastructure systems. Most of them
  are introduced during the development process as a~result of hidden errors, which might not appear
  suspicious initially. The system and its users or their data are at risk until the flaws are patched.
  That is the main reason and motivation why it is important to constantly improve the security of products.

  Cloned vulnerabilities are security weaknesses that are introduced into the~software system
  when code is copied or reused from another system that contains the vulnerability.
  These vulnerabilities can be difficult to detect and fix because they are not necessarily introduced
  by intention, instead, they are inherited from the source code that was copied or reused. In software
  engineering, the approach of cloning similar functional parts already implemented in other applications
  is usually applied. It makes the development of new products or adding features to existing ones swifter.

  Cryptocurrencies, which became very popular in recent years, are a good example of this case.
  Namely, Bitcoin, an Open-Source peer-to-peer electronic cash system created by Satoshi Nakamoto~\cite{bitcoin}
  inspired many new projects that joined the cryptocurrency market. Lots of them were created
  as derivatives of Bitcoin with the idea to extend or improve its features. Cloning helped to speed up
  the development of new coins by inheriting its base infrastructure.

  Although, neither a~large-scale project developed by the community as Bitcoin is always perfect. Plenty of
  vulnerabilities were discovered in its code base which were accordingly documented and are stored
  and tracked in vulnerability databases. As there are many other coins that share its code, it is possible
  that they also share the same vulnerabilities. The question inspired this work to develop a~monitoring tool
  with the goal of to analyse the~threat and help with the detection of vulnerable code and its occurrence in
  cloned projects, as the identification is not an easy but rather costly and exhaustive process
  and after identification yet also patching the issue is desired.

  The prevalence of code reuse and the increasing complexity of software systems makes cloned vulnerabilities
  an important issue to consider in software development and maintenance. This thesis aims to study the
  characteristics and impacts of cloned vulnerabilities and to identify effective approaches for detecting
  and mitigating them. The proposed tool in this work considers disclosed vulnerabilities which means
  that the issue was already patched in the project that was originally affected by it. Thanks to this fact
  the tool can identify an issue, the affected code in the original project, and candidate projects with
  the~probability of vulnerability inheritance. Additionally, the tool can be configured to run in schedules
  and identify potential bugs in the monitored project. The identified bugs become candidates for
  detection of their adoption in projects forked from the monitored project.

  An existing tool, with the same goal described above, was implemented in a~project named CoinWatch~\cite{CoinWatch}
  with an aim at vulnerabilities in cryptocurrencies. The CoinWatch inspired this work
  with an idea to bring improvements, extensions, and a~graphical user interface for wider and simplified
  usage of the tool for detecting and mitigating cloned vulnerabilities.

  This thesis begins with a~basic introduction to the problem and the motivation for why it is relevant
  to deal with. Chapter \ref{chapter:vulnerabilities} explains and takes a~closer look at vulnerabilities
  and the basic terminology connected with them. In Chapter \ref{chapter:clonedVulnerabilities}, clones of source
  code, current detection tools and approaches are described and analyzed. Afterwards, Chapter \ref{chapter:design}
  describes a~draft of the tool built for detecting cloned vulnerabilities. The next two Chapters
  \ref{chapter:implementation} and~\ref{chapter:experimentation} contain implementation details
  and an evaluation of the developed product. The final Chapter \ref{chapter:conclusion} concludes this work
  with potential improvements for future work.
