% File: src/chapter_6_experimentation.tex
% Project: Monitoring and Reporting Tool for Cloned Vulnerabilities across Open-Source Projects
% Author: Matus Remen (xremen01@stud.fit.vutbr.cz)
% Description: Chapter 6 - Experimentation

\newcommand{\stimes}{\ensuremath{\times}}

\chapter{Experimentation}
\label{chapter:experimentation}
This chapter presents and evaluates the capabilities of the implemented tool. The first section describes the preparation steps,
including system configuration and data set. Then the results produced by the implemented tool are presented in the following
section. Finally, the last section summarizes the results, discusses shortcomings and provides suggestions for possible
improvements and next development.

\section{Preparation}
Initially, the open-source projects \emph{Bitcoin} and \emph{Go-Ethereum} were selected for experiments because of their
popularity, according to the list of cryptocurrencies from the website \emph{CoinGecko}\footnote{\href{https://www.coingecko.com}
{https://www.coingecko.com}}.
The projects were then registered in the internal database as parent projects and their repositories were cloned to the local
storage. From the previously mentioned list of cryptocurrencies and based on prior research, projects which adopted source
code from either \emph{Bitcoin} or \emph{Go-Ethereum} repositories were identified. Accordingly, the identified projects were
registered in the internal database as their forks and cloned to the local storage.

The data set used for experimentation consists of vulnerabilities with assigned CVE identifiers, discovered in the selected parent
projects \emph{Bitcon} and \emph{Go-Ethereum}. The vulnerabilities were selected based on the availability of information about
them, mainly references to patches in order to work with verified data. Accordingly, the web interface was used to create
entities of the bugs in the internal database and attempt to find bug-fixing candidate commits using the functionality of the
component \texttt{FixCommitFinder}. The found candidate commits were then manually validated and the bug-fixing code changes were
extracted, in order to use in the detection method only changes that address patch of the vulnerability. The extracted code
changes were then used to update the corresponding attributes of the bug entity in the internal database, in order to allow
this step to be omitted in the subsequent repeated executions of the detection.

The docker environment was configured to use 4~CPU cores to take advantage of implemented multiprocessing features which
improve the speed of the algorithm and 8~GB of memory for the experimentation. For particular evaluations, the web interface
was used to execute both detection methods for each vulnerability from the prepared data set. Consequently, the detection
results were noted and manually verified in the corresponding code bases in order to evaluate the precision of the implemented
detection methods. Manual verification is needed because the results might contain false detections.

\section{Results}
This section presents the detection results for the prepared list of vulnerabilities. Selected vulnerabilities are closely
analysed to address the capabilities and shortcomings of particular implemented detection methods, while the others provide
only numbers of positive/negative results for the calculation of the success rate.
The detailed results are compared to the expected outcome of the tool and are presented in tables, where a check mark (\checkmark)
represents detected patch, a cross (\stimes) represents that patch was not applied but the vulnerable code detected and an empty
cell means that the clone was not found and vulnerability was not propagated. The results, which do not correspond to reality
(false positives/negatives) are marked by the red color of the cell. The tables display results from the main detection method,
which is based on the approach of BlockScope.

\subsection*{Bitcoin-based vulnerabilities}
A~vulnerability with identifier CVE-2021-3401\footnote{\href{https://nvd.nist.gov/vuln/detail/CVE-2021-3401}
{https://nvd.nist.gov/vuln/detail/CVE-2021-3401}} was selected for the first experiment, as it is the latest published
vulnerability according to the list\footnote{\href{https://en.bitcoin.it/wiki/Common_Vulnerabilities_and_Exposures}
{https://en.bitcoin.it/wiki/Common\_Vulnerabilities\_and\_Exposures}} of weaknesses in \emph{Bitcoin}. The vulnerability was discovered
in the project \emph{Bitcoin} and might allow an attacker to execute arbitrary code upon passing a~malicious argument
to the \texttt{bitcoin-qt} program. This was caused by misuse of built-in arguments of GUI framework Qt\footnote{\href{
https://www.qt.io}{https://www.qt.io}}.

% BTC sv, zen - clones of zcash
\begin{table}[h]
  \centering
  \begin{tabular}{|l|c|c|c|}
    \hline
    CVE-2021-3401 & \multicolumn{3}{|c|}{Patch application status over date} \\
    \hline
    \multicolumn{1}{|c|}{Project} & 21/02/04\ & 22/05/06\ & 23/05/03 \\
    \hline
    bitcoin-abc & \checkmark & \checkmark & \checkmark \\  % T III & T III & T III
    \hline
    bitcoin-sv & & & \\
    \hline
    BTCGPU & \stimes & \stimes & \stimes \\  % T I & T I & T I
    \hline
    dash & \stimes & \checkmark & \checkmark \\  % T I & T II & T II
    \hline
    dogecoin & \stimes & \checkmark & \checkmark \\  % T III & T II & T II
    \hline
    litecoin & \checkmark & \checkmark & \checkmark \\  % T I & T II & T II
    \hline
    pigeoncoin & \stimes & \stimes & \stimes \\  % T III & T III & T III
    \hline
    Ravencoin & \stimes & \stimes & \checkmark \\  % T III & T III & T II
    \hline
    qtum & \checkmark & \checkmark & \cellcolor{red!25} \\  % T I & T I & T II
    \hline
    zcash & & & \\
    \hline
    zen & & & \\
    \hline
  \end{tabular}
  \caption{Detection results for CVE-2021-3401 over various versions of the analysed projects.}
  \label{tab:results-cve-3401}
\end{table}

The results in Table~\ref{tab:results-cve-3401} show that the vulnerability was propagated to some clones of the project Bitcoin.
It contains patch application status for three different states of the projects according to the timeline of each project.
The first, the 4th of February 2021 refers to the date of CVE publishing, and the following display patching progress over time
with the last date, the 3rd of May 2023, referring to the date of this experiment. The results contain the detection of the first
three types of clones -- Type~I, Type~II and Type~III.

Projects \emph{bitcoin-abc}, \emph{dogecoin}, \emph{pigeoncoin} and \emph{Ravencoin} contained clones of Type~III. Clones of
Type~II were found in projects \emph{dash}, \emph{dogecoin} and \emph{Ravencoin} after applying patch and the rest were Type~I
clones. Projects \emph{bitcoin-sv}, \emph{zcash} and \emph{zen} did not adopt the vulnerable code. Although, there was one
miss in case of the project \emph{qtum} in the most recent version, where the upper candidate context was missing, thus the
candidate code was not identified, even when the lower context would be found.

The tool Simian was able to detect vulnerability only in the project \emph{BTCGPU} containing Type~I clone, when run with code
fragment containing the vulnerable version of the code on version from 3rd of May 2023. With version from 4th of February 2021
it was additionally able to detect the vulnerability only in project \emph{dash}, still missing the other three affected projects.

% 23/05/03
% valid - BTCgold - T I
% valid - doge, dash, litecoin, raven-qt - T II
% valid - pigeon, BTCabc,  - T III missing params, reordered params
% invalid - qtum - upper ctx diverged, patch applied T II - qtum-qt
% 23/05/03 BTCsv, zcash, zen -- no Qt at all
% Simian - detected only BTCGPU
% Simian - no support for Go

% 22/05/06
% valid BTCabc, BTCgold, doge, dash, litecoin, pigeon, qtum
% valid Raven - T II
% SIMIAN detected only BTCGPU

% 21/02/04
% digi T II valid - not applied

\begin{table}[h]
  \centering
  \begin{tabular}{|l|c|c|c|}
    \hline
    CVE-2018-17144 & \multicolumn{3}{|c|}{Patch application status over date} \\
    \hline
    \multicolumn{1}{|c|}{Project} & 18/09/14 & 18/09/19 & 23/05/03 \\
    \hline
    bitcoin-abc & & &  \\  % no clone
    \hline
    bitcoin-sv & & & \\  % no clone
    \hline
    BTCGPU & \stimes & \checkmark & \checkmark \\  % T I & & T I
    \hline
    dash & \cellcolor{red!25} &  & \checkmark \\  % T I & T III/IV? & T III fixed?
    \hline
    dogecoin & \stimes & \checkmark & \checkmark \\  % T I & & T I & T I
    %\hline
    %digibyte & \stimes & \stimes & \checkmark \\  % T I
    \hline
    litecoin & \stimes & \checkmark &  \\  % T I & T I
    \hline
    pigeoncoin & \cellcolor{red!25} & \cellcolor{red!25} & \cellcolor{red!25}\stimes \\  % T I
    \hline
    Ravencoin & \stimes & \checkmark &  \\  % T III & T III &
    \hline
    qtum & \cellcolor{red!25} & \checkmark & \\  % T I & T I
    \hline
    zcash & \cellcolor{red!25} & & \\  % T III
    \hline
    zen & \cellcolor{red!25} & & \\  % T III
    \hline
  \end{tabular}
  \caption{Detection results for CVE-2018-17144 over various versions of the analysed projects.}
  \label{tab:results-cve-17144}
\end{table}

An \emph{Inflation bug} was chosen for the second detailed experiment. The bug was discovered alongside Denial-of-Service weakness
in \emph{Bitcoin} and was described in Section~\ref{sec:example-cloned-vuln}. The vulnerability was assigned the identifier
CVE-2018-17144\footnote{\href{https://nvd.nist.gov/vuln/detail/CVE-2018-17144}{https://nvd.nist.gov/vuln/detail/CVE-2018-17144}}.
The experiments were done with various versions of the projects and the most interesting are contained
in Table~\ref{tab:results-cve-17144}. The dates in the table refer to the date before publication of the CVE record, the date
of publication and the date of experimentation.

The results of the tool on this vulnerability are worse in comparison to the previous vulnerability. The reason is that the projects where
the clone was not detected used a version where the context of the vulnerable code contained too many adjustments (\texttt{d1dee20547}) despite the affected code being present. Simian detected additionally
the flaw in projects \emph{dash} and \emph{pigeoncoin}, although missed \emph{Ravencoin} which contained a clone of Type~III
on the 14th of September 2018. On the other side, projects \emph{bitcoin-abc} and \emph{bitcoin-sv} did not clone the affected
code.

To the 3rd of May 2023, most of the repositories already patched the vulnerability using different solutions, thus syntactical
clones were mostly not found, which makes the original patch outdated. Additionally, the original patch contained a~change only on
a single line of code\footnote{\href{https://github.com/bitcoin/bitcoin/pull/14249/commits/d1dee20547}
{https://github.com/bitcoin/bitcoin/pull/14249/commits/d1dee20547}}, specifically changing
value of parameter from \texttt{true} to \texttt{false}. In this case, the similarity-based method would require more strict rules, to pay closer attention to the specific changes.
The lack of precision resulted in a false negative detection in project \emph{pigeoncoin}, as the vulnerability was
patched here using different logic. Extended context was used also in the case of \emph{qtum}, \emph{zcash}
and \emph{zen}.

\begin{table}[h]
    \centering
    \begin{tabular}{|c|c|c|c|c|c|c|}
         \hline
         Vulnerability & Method             & TP & TN & FP & FN & Date \\
         \hline
         \multirow{4}{*}{CVE-2019-15947} &BS & 4 & 7 & 1 & & \multirow{2}{*}{23/05/03} \\  % T Is - FP upper context modified
                                         &SA & 4 & 8 &   & & \\%23/05/03 \\  % T Is
                                         \cline{2-7}
                                         &BS & 8 & 2 &   & & \multirow{2}{*}{20/04/01} \\
                                         &SA & 8 & 4 &   & & \\ %20/04/01 \\
                                         \hline
         \multirow{4}{*}{CVE-2018-17145} &BS & 1 & 7 &   & & \multirow{2}{*}{23/05/03} \\  % T III png
                                         &SA &   & 11  &   & 1 & \\ %23/05/03 \\
                                         \cline{2-7}
                                         &BS & 2 & 5 &   & 1 & \multirow{2}{*}{20/04/01} \\ % T III rvn & pgn, FP dash missing missing UP
                                         &SA & 1 & 9 &   & 2 & \\%20/04/01 \\ % FN rvn & png
         \hline
    \end{tabular}
    \caption{Evaluation of both implemented detection methods on CVE-2019-15947 and CVE-2018-17145.}
    \label{tab:results-quant}
\end{table}

Summarized detection results comparing the two implemented detection methods for another two vulnerabilities are contained in
Table~\ref{tab:results-quant}. Column \emph{TP} represents true positive detections -- the vulnerable code was correctly detected,
column \emph{TN} represents true negative results -- repository does not contain vulnerable code, and columns \emph{FP} and \emph{FN} represent
corresponding false results. The particular methods are marked as \emph{BS}, which refers to the BlockScope-based method, and \emph{SA}
which stands for Simian or similarity analyser.

In the case of vulnerability CVE-2019-15947, as all the detection were clones of Type~I, the tool Simian was able to perform
slightly better. The one false positive result of the BlockScope-based method was caused by stretched candidate code because
of the upper candidate context, which was not precisely matched.
On the other hand, the code of vulnerability CVE-2018-17145 contained clones of Type~III as well in the forked repositories,
which were not detected by Simian, but by BlockScope were.

\subsection*{Go-Ethereum-based vulnerabilities}
The third closely analysed vulnerability was present in project \emph{Go-Ethereum}, which is implemented in the programming language Go
in contrary to previously analysed projects written in the programming language C++. The selected vulnerability was assigned an identification
CVE-2022-29177\footnote{\href{https://nvd.nist.gov/vuln/detail/CVE-2022-29177}{https://nvd.nist.gov/vuln/detail/CVE-2022-29177}}
and its exploitation could make the affected node crash. It was selected because it is the latest vulnerability in this project,
which has a reference to the patch and is an example of complex changes discussed in relation to the discovery scan.
The discovery scan would evaluate it as complex because the patch affects more than one file.

The detection results can be found in Table~\ref{tab:results-cve-29177}.
The first date corresponds to the fix in \emph{Go-Ethereum}, the second date indicates the CVE publication date, and the final date represents the date of this experiment.
This vulnerability was propagated to four of five analysed projects as a clone of Type~I and the main detection method
was able to detect it correctly, while Simian does not support this language, it needed to be configured for a plain text
comparison.

% bor - MATIC - Polygon
% bsc - BNB - Binance
% celo - CELO
% subnet-evm - AVAX - Avalanche
% optimism - OP
\begin{table}[h]
  \centering
  \begin{tabular}{|l|c|c|c|}
    \hline
    CVE-2022-29177 & \multicolumn{3}{|c|}{Patch application status over date} \\
    \hline
    \multicolumn{1}{|c|}{Project} & 22/03/07 & 22/06/20 & 23/05/03 \\
    \hline
    bor & \stimes & \checkmark &  \checkmark \\  % T I, T I, T I
    \hline
    bsc & \stimes & \checkmark & \checkmark \\  % T I, T I, T I
    \hline
    celo-blockchain & \stimes & \checkmark & \checkmark \\  % T I, T I, T I
    \hline
    optimism & \stimes & \stimes & \stimes \\  % T I, T I, T I
    \hline
    subnet-evm & & &  \\  % no clone
    \hline
  \end{tabular}
  \caption{Detection results for CVE-2022-29177 over various versions of the analysed projects.}
  \label{tab:results-cve-29177}
\end{table}

\vspace{-1em}

\section{Evaluation}
The detection during experimentation was executed using the implemented web interface, which was used initially to prepare
repositories and the data set. The data set consisted of five vulnerabilities discovered in either project \emph{Bitcoin}
or \emph{Go-Ethereum}, which covered all three types of patches -- containing only additions, only deletions, and mixed changes.

Twelve projects which adopted code from \emph{Bitcoin} and five from project \emph{Go-Ethereum} were selected for experiments.
The experiments were designed to address each vulnerability in the data set on various versions from the timeline of the forked
projects, which was easily possible thanks to the optional parameter specifying the version date in the forked projects.
Consequently, both available detection methods were executed and the BlockScope-based method was able to detect also clones
of Type~II and Type~III additionally to the clones of Type~I detected by the integrated tool Simian as well.

Although, the experimentation confirmed the expected shortcomings and advantages of detection methods. Simian is limited to
the detection of Type~I clones which generated false negative detection results. The higher types of clones were covered
by the second method, \emph{BlockScope}, utilizing textual context-based candidate code search and textual similarity-based comparison with a patch
code for determining the vulnerability of candidate code fragments. The second method would fail at finding the right candidate
context in the target project or due to using a relatively low threshold for the similarity between patches with minor changes
and candidate code. The experiments resulted in two false positives and one false negative result on the prepared data set
and over various versions from the timeline of forked projects. The two false-positive results were identified on the date
of the experiment, which was employed for detecting all vulnerabilities. Consequently, this date was chosen for the
calculation, resulting in an 80\% true positive rate of the implemented detection method covering the first three clone types.

The possible improvements could be achieved by defining stricter rules for patches containing specific changes as it
was in the analysis of CVE-2018-17144, where the initial patch changed only the boolean value in the function call.
The Normalized Levenshtein edit distance metric~\cite{strsim} evaluated the vulnerable boolean value \texttt{false} more
similar to the patched parameter with the boolean value in project \emph{pigeoncoin}, resulting in false positive detection.
Additional extensions for the implemented tool could contain support for more file extensions to the current \texttt{.cpp}
and \texttt{.go}, which currently helps with filtering files and comment lines.
