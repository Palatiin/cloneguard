
\chapter{Introduction}
  Cloned vulnerabilities are security weaknesses that are introduced into a software system
  when code is copied or reused from another system that contains the vulnerability.
  These vulnerabilities can be difficult to detect and fix, because they are not necessarily introduced
  intentionally, instead, they are inherited from the source code that was copied or reused. It is common
  in software engineering to clone similar functional parts already created in~other applications when
  developing new products or adding features to existing ones.

  For this case, a good example is cryptocurrencies which became very popular in recent years.
  Namely, Bitcoin, an Open-Source peer-to-peer electronic cash system created by Satoshi Nakamoto
  \cite{bitcoin} inspired many new projects that joined the cryptocurrency market. Lot~of them were created
  as derivatives of Bitcoin with the idea to extend or improve its features and cloning helped to speed up
  the development of new coins by inheriting its base infrastructure.

  Although, neither a project as phenomenal as Bitcoin is always perfect, plenty of vulnerabilities were
  discovered in its code base which were accordingly documented and are stored and tracked in vulnerability
  databases. As there are many other coins that share its code, it is possible that they also share
  the same vulnerabilities. The question inspired this work to develop a monitoring tool with the goal
  to analyze the threat and help with the detection of the vulnerable code and its occurrence in
  cloned projects, as the identification is not an easy but rather costly and exhaustive process
  and after identification yet also patching the issue is desired.

  The prevalence of code reuse and the increasing complexity of software systems make cloned vulnerabilities
  an important issue to consider in software development and maintenance. This thesis aims to study the
  characteristics and impacts of cloned vulnerabilities and to identify effective approaches for detecting
  and mitigating them. The proposed tool in this work considers disclosed vulnerabilities which means
  that the issue was already patched in the project that was originally affected by it. Thanks to this fact
  the tool can identify an issue, the affected code in the original project, and candidate projects with
  the probability of vulnerability inheritance and additionally the tool can propose also a patch for
  the detected threat.

  An existing tool, with the same goal described above, was implemented in a project named CoinWatch
  \cite{CoinWatch} with an aim on vulnerabilities in cryptocurrencies. The CoinWatch inspired this work
  with an idea to bring improvements, extensions, and a graphical user interface for wider and simplified
  usage of the tool for detecting and mitigating cloned vulnerabilities.


\chapter{Vulnerabilities in Software Applications}
  Software vulnerabilities and exposures are weaknesses or flaws in software products that can be
  exploitable by an attacker. By exploitation of a vulnerability, an attacker might be able to gain
  an~unauthorized access or perform malicious actions which can affect either the targeted application
  or~its users.

  Most of the known vulnerabilities are associated with dealing with input provided by a user
  of the application. For instance, some frequent types of vulnerabilities belong buffer overflows,
  cross site scripting, and SQL injections.\,\cite{vulnerabilities} On the other hand, some of weaknesses
  might be also caused by using insecure libraries or frameworks, but mostly during the development process
  through coding mistakes, so-called bugs.

  The consequences of vulnerabilities in software applications can be quite serious like a data breaches,
  theft of sensitive information, or damage to a product infrastructure. For instance, consider
  vulnerability, in Microsoft Windows implementation of Server Message Block protocol, with an identifier
  CVE-2017-0144. An exploitation of this flaw, by sending crafted packets, allows remote attackers to run
  arbitrary code on a target machine. This defect facilitated the spreading of \emph{WannaCry} ransomware
  through the network in~2017, which affected many organizations, companies, and individuals.
  % https://nvd.nist.gov/vuln/detail/CVE-2017-0144, https://www.zdnet.com/article/ransomware-an-executive-guide-to-one-of-the-biggest-menaces-on-the-web/

  Ransomware is a type of malicious software, that locks up the victim's data or device and threatens to delete
  or keep it locked unless a ransom is paid to an attacker. In the case of \emph{WannaCry}, the malware
  would encrypt files on the victim's device and ask for a ransom of value 300 USD in Bitcoin if paid within
  the first three days, otherwise, the value would be doubled for the next four days and if not paid at all,
  the files would be lost forever.
  % https://www.ibm.com/topics/ransomware, https://www.csoonline.com/article/3227906/wannacry-explained-a-perfect-ransomware-storm.html

  % Figure

  In order to mitigate repeated occurrences and time consuming process of patching, each
  vulnerability and exposure is documented and stored in a database, so repeated mistakes in implementation
  can be mitigated or fixed faster. This source provides valuable information for the public,
  security engineers, and for tools working with weaknesses in software applications to improve
  cyber security.

  \section{Vulnerability Record Notations}
    Each record in the database must follow a schema, which holds a number of mandatory and optional fields
    describing the vulnerability.
    \subsection*{Common Vulnerabilities and Exposures - CVE}
      \paragraph{CVE Fields Description}
    \subsection*{CPE and Other (TODO)}

  \section{Vulnerability Database}
    Access via web page or API, organizations, maintainers, creating a new report.

\newpage


\chapter{Detection Tools and Approaches}

  \section{Web Scraping and Vulnerability Parsing}

  \section{Code Evolution Analysis - SZZ Algorithm}

  \section{Identification of Vulnerable Code}

  \section{Clone Detection}
    \subsection*{Types of Clones}
      There are three types of clones we register nowadays. Depending on  how similar they are to their
      origins, they are divided into groups:
      \paragraph{Type I. Clones}
      \paragraph{Type II. Clones}
      \paragraph{Type III. Clones}

  \section{Existing Tools}

    \subsection*{CoinWatch}
      Introduction (Purpose, ...) \\
      Workflow, references

    \subsection*{Other tools}


\chapter{Design}


\chapter{Implementation}


\chapter{Experimentation}


\chapter{Conclusion}
